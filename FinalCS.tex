\documentclass{article}
\usepackage[a4paper,margin=0mm]{geometry}
\usepackage{parskip,setspace,titlesec,enumitem,tcolorbox,amsmath,amssymb}

\setstretch{0.6}
\setlength{\parindent}{0pt}
\setlength{\parskip}{0pt}
\titlespacing*{\section}{0pt}{0pt}{0pt}
\setlist{nosep,leftmargin=3mm}

\begin{document}
\raggedbottom
\scriptsize
\begin{tcolorbox}[colframe=black,colback=white,boxrule=0.3pt,arc=1pt,
left=0pt,right=0pt,top=0pt,bottom=0pt]

\begin{minipage}[t]{0.49\textwidth}
\textbf{\S1.1 Error Analysis}
\begin{itemize}
\item \textbf{Normalization}: 32 bit - Sign bit: 1, Sign exp: 1, Exp: 7, Normalized mantissa: 23.
\item Absolute Error = $|p-p^*|$, Relative Error = $\frac{|p-p^*|}{|p|}$
\item Significant Digits: RE $<5\times10^{-t}$
\item \textbf{fl(x)}: Machine representation
\item \textbf{Cancellation Error}: Subtracting nearly equal numbers
\item \textbf{Example}: $p=0.54617$, $q=0.54601$, true $r=p-q=0.00016$\\ 
4-digit: $p^*=0.5462$, $q^*=0.5460$, $r^*=0.002$ (RE=25\%)
\item \textbf{Nested Multiplication}: Reduces error\\ 
$f(z)=1.01z^4-4.62z^3-3.11z^2+12.2z-1.99=(((1.01z-4.62)z-3.11)z+12.2)z-1.99$
\end{itemize}

\begin{itemize}
\item $P_n(x)=\sum_{k=0}^n\frac{f^{(k)}(a)}{k!}(x-a)^k$
\item Remainder: $R_n(x)=\frac{f^{(n+1)}(\xi)}{(n+1)!}(x-a)^{n+1}$, $\xi\in(a,x)$
\item Linear Approx: $f(x_0+h)\approx f(x_0)+hf'(x_0)$
\item \textbf{Example}: $\xi\in(0,\pi/2)$, $\sin\xi\leq1 \Rightarrow R_n\leq\frac{(x)^n}{n!}$
\end{itemize}

\textbf{\S1.3 Convergence}
\begin{itemize}
\item $\alpha=\lim_{n\rightarrow\infty}(\alpha_n)$
\item Rate: $\alpha_n = \alpha + \mathcal{O}(\beta_n)$ if $|\alpha_n-\alpha|\leq K|\beta_n|$
\item Find largest $p$ where $\alpha_n-\alpha=\mathcal{O}(1/n^p)$
\end{itemize}

\textbf{\S1.4 Matrix Operations}
\begin{itemize}
\item $A = \begin{bmatrix}a_{11}&a_{12}&a_{13}\\a_{21}&a_{22}&a_{23}\end{bmatrix}$, 
$B = \begin{bmatrix}b_{11}&b_{12}\\b_{21}&b_{22}\\b_{31}&b_{32}\end{bmatrix}$
\item $AB = \begin{bmatrix}
a_{11}b_{11}+a_{12}b_{21}+a_{13}b_{31} & a_{11}b_{12}+a_{12}b_{22}+a_{13}b_{32}\\
a_{21}b_{11}+a_{22}b_{21}+a_{23}b_{31} & a_{21}b_{12}+a_{22}b_{22}+a_{23}b_{32}
\end{bmatrix}$
\end{itemize}

\textbf{\S6.1 Gaussian Elimination}
\begin{itemize}
\item $\mathcal{O}(n^3)$ complexity
\item \textbf{Pivoting}:
\begin{itemize}
\item Partial (PP): Max element in column
\item Scaled PP: $s_i=\max_j|a_{ij}|$, pivot $\max(a_{ik}/s_i)$
\item Complete (CP): Full matrix search ($\mathcal{O}(n^3)$)
\end{itemize}
\item \textbf{LU Decomposition}: $PA=LU$ through GE steps and $LUx = Pb$.
\item \textbf{LU Algorithm}: $L = E_{n-a,m-b}^{-1}E_{n-a+1,m-b+1}^{-1}\cdots$ $U = $ GE.
\item \textbf{Choleski Factorization}: If a matrix is symmetric and positive definite, it may be factored to the form $LDL^T$
\end{itemize}

\textbf{\S6.2 Special Matrices}
\begin{itemize}
\item \textbf{Inverse Matrix}: An inverse matrix of A is $A^{-1}$ such that $AA^{-1}=I$
\item Properties: $(AB)^{-1}=B^{-1}A^{-1}$, $(A^{-1})^T=(A^T)^{-1}$
\item \textbf{Singular}: A matrix is singular iff its det is 0.
\item \textbf{Diagonal}: $d_{ij}=0$ for $i\neq j$: All non-diagonal entries are 0.
\item \textbf{Symmetric}: $A=A^T$, $(AB)^T=B^TA^T$
\item \textbf{Permutation}: Row swaps of $I_n$, $PA$ reorders rows: $P^T = P^{-1}$
\item \textbf{Diagonally Dominant}: $|a_{ii}| > \sum_{j\neq i}|a_{ij}|$ (nonsingular)
\item \textbf{Positive Definite}: $x^TAx>0$ $\Rightarrow$ $A=LDL^T$, $a_{ii}>0$, $a_{ij}^2<a_{ii}a_{jj}$
\item \textbf{Minor}: $M_{ij}$ is a submatrix of A with the row i deleted and column j deleted.
\item \textbf{Band}: an $n\times n$ matrix is a band matrix if $p,q\in \mathbb{Z}:1\le p, q\le n$ exist with $a_{i_j} = 0$ for $i+p\le j$ or $j+q\ge i$ The bandwidth is defined as $w=p+q-1$. For a diagonal matrix, $p=1,q=1,w=1$
\item \textbf{Tridiagonal}: Band with $p=2, q=2$. It exhibits the following properties
\begin{minipage}[t]{0.45\textwidth}
\begin{itemize}
    \item $a_{ii} = l_{ii}$
    \item $a_{i,i-1}=l_{i,i-1}: i=2,3,\cdots,n$
    \end{itemize}
    \end{minipage}
\begin{minipage}[t]{0.45\textwidth}
    \begin{itemize}
    \item $a_{i,i+1}=l_{ii}u_{i,i+1}:i=1\cdots n-1$
    \item $a_{ii} = l_{i,i-1}u_{i-1,i}+l_{ii}: i=2\cdots n$
\end{itemize}
\end{minipage}
\item Crout Factorization: This factorization happens in $O(n)$ time
\end{itemize}

\textbf{\S Strategies}
\begin{itemize}
    \item \textbf{RoC With inf limit}: set $h=1/n$ and solve accordingly. 
    \item \textbf{$D\cdot (L+U)$}: given D has ONLY diagonal entries and L+U has NO diagonal entries, the resulting matrix A is composed of entries $a_{ij}=d_{ii}\cdot (l+u)_{ij}$
    \item \textbf{Verification of Bisection}: To verify bisection can be applied, make sure that f(a) and f(b) are of different signs.
    \item \textbf{Error of Bisection}: To compute the accuracy of bisection to an $\varepsilon$, we use $\frac{b-a}{2^n}\le 
    \varepsilon$
    \item \textbf{Failure of Newton's Method}: NM Fails if $f'(x)=0$ for some x.
    \item \textbf{Triangle Inequality}: $|x+y|\le |x|+|y|$
\end{itemize}

\textbf{\S Key Definitions \& Identities}
\begin{itemize}
    \item \textbf{Continuity}: $f\in C^n[a,b]$ reads: the nth derivative of f on [a,b] is continuous.
    \item \textbf{Series Expansions}
    \begin{itemize}
        \item $e^x = 1 + x + \frac{x^2}{2!} + \frac{x^3}{3!} + \cdots \quad (\forall x \in \mathbb{R})$
        \item $\sin x = x - \frac{x^3}{3!} + \frac{x^5}{5!} - \frac{x^7}{7!} + \cdots$
        \item $\cos x = 1 - \frac{x^2}{2!} + \frac{x^4}{4!} - \frac{x^6}{6!} + \cdots$
        \item $\arctan x = x - \frac{x^3}{3} + \frac{x^5}{5} - \frac{x^7}{7} + \cdots \quad (|x| \leq 1)$
        \item $\ln(1+x) = x - \frac{x^2}{2} + \frac{x^3}{3} - \frac{x^4}{4} + \cdots \quad (|x| < 1)$
        \item $\frac{1}{1-x}=1+x+x^2+x^3+x^4+\cdots: |x|<1$
        \item $\frac{1}{1+x}=1-x+x^2-x^3+x^4-\cdots: |x|<1$
    \end{itemize}
    
\begin{minipage}[t]{0.45\textwidth}
    \textbf{Core Identities}
    \begin{itemize}
        \item $\sin^2\theta + \cos^2\theta = 1$
        \item $\sin(a\pm b) = \sin a\cos b \pm \cos a\sin b$
        \item $\cos(a\pm b) = \cos a\cos b \mp \sin a\sin b$
    \end{itemize}
\end{minipage}
\hfill
\begin{minipage}[t]{0.45\textwidth}
    \textbf{Angle Transformations}
    \begin{itemize}
        \item $\sin 2\theta = 2\sin\theta\cos\theta$
        \item $\cos 2\theta = 2\cos^2\theta - 1 = 1 - 2\sin^2\theta$
        \item $\sin^2\theta = \frac{1-\cos2\theta}{2}$, $\cos^2\theta = \frac{1+\cos2\theta}{2}$
    \end{itemize}
\end{minipage}

    \item \textbf{Vector Norms}\\
    \begin{minipage}[t]{0.45\textwidth}
    \begin{itemize}
       \item $||x||>0$
       \item $||x|| = 0\Leftrightarrow x=0$
    \end{itemize}
    \end{minipage}
    \hfill
    \begin{minipage}[t]{0.45\textwidth}
    \begin{itemize}
       \item $||\alpha x|| = |\alpha| ||x||$
       \item $||x+y||\le ||x|| + ||y||$
    \end{itemize}
    \end{minipage}
\end{itemize}
\textbf{\S Useful Examples}
\begin{itemize}
    \item Suppose $g(x)=\frac{5}{x^2}+2$ Show $p_n=g(p_{n-1})$ will converge to g for $\forall p_0[2.5,3]$.\\
    Since this is a decreasing function, the max of g(x) is g(2.5) and the min of g(x) is g(3).
    \begin{itemize}
        \item First compute the max, $g(2.5)=14/5 < 3$
        \item Second compute the min, $g(3)=5/9 + 2>2.5$
        \item Last compute $|g'(x)|=-10/x^3 \le max_{x\in[2.5,3]}|g'(x)|=16/25<1$
    \end{itemize}
    \item Given $||A||$ is a natural matrix norm of matrix A. show $|\lambda|\le||A||$ for any nonsingular A and any $\lambda$ of A.
    $||A||=\max_{||x||=1}||Ax|| \ge ||Ax||:x \text{ is an e-vec s.t ||x||=1}=||\lambda x|| = |\lambda|||x||=|\lambda|\square$
    \item When performing Jacobi or GS, when computing L+U, flip the signs of all entries.
    \item To determine convergence for fixed point, compute $g'(p_0)\le1$, which gives a,b. Prove $g(x)$ cts on [a,b], $g(x)\in[a,b]$, $g'(x)$ exists on (a,b), $|g'(x)|\le k:\forall x \in(a,b), 0<k<1$
\end{itemize}
\end{minipage}
\hfill
\begin{minipage}[t]{0.49\textwidth}
\textbf{\S LA Determinants}
\begin{itemize}
\item 2x2: $|A|=ad-bc$
\item nxn: $|A|=\sum_{j=1}^n a_{ij}A_{ij}$ via cofactors $A_{ij}=(-1)^{i+j}M_{ij}$
\begin{minipage}[t]{0.45\textwidth}
\textbf{Properties}
    \begin{itemize}
        \item Swap rows: $|\tilde{A}| = -|A|$
        \item Scale row: $|\tilde{A}| = \lambda |A|$
    \end{itemize}
\end{minipage}
\hfill
\begin{minipage}[t]{0.45\textwidth}
    \begin{itemize}
        \item Identical rows: $|A| = 0$
        \item $|AB| = |A||B|$, \quad $|A^T| = |A|$
        \item $|A^{-1}| = \frac{1}{|A|}$
    \end{itemize}
\end{minipage}
\end{itemize}
\textbf{\S7 Norms \& Eigen}\\
\begin{minipage}[t]{0.45\textwidth}
\begin{itemize}
\item $||x||_2=\sqrt{\sum x_i^2}$, $||x||_\infty=\max|x_i|$
\item $||A||_\infty=\max_i\sum_j|a_{ij}|$ Basically sum all rows together and determine the largest one.
\end{itemize}
\end{minipage}
\begin{minipage}[t]{0.45\textwidth}
\begin{itemize}
\item $||A||_2 = \sqrt{\rho(A)}$
\item $||A|| = \max_{||x||=1}||Ax||$
\end{itemize}
\end{minipage}
\begin{itemize}
\item $\forall x \in \mathbb{R}^n: ||x||_\infty \le ||x||_2 \le \sqrt{n}||x||_\infty$
\item A distance between matrices A and B wrt a matrix norm $||\cdot||$ is $||A-B||$
\item Theorem: For any vector $x\ne0$, matrix Am and abt natural norm $||\cdot||$ we have $||Ax|| \le ||A||\cdot||x||$
\item Cauchy-Schwarz: $||x+y||_2\leq||x||_2||y||_2$
\item Eigen: $\lambda$ is an eigenvalue if $A\lambda=v \cdot \lambda$
\item Finding Eigenthings: $\det (A-\lambda I)=0:\forall \lambda$.
\item Spectral Radius: $\rho(A)=\max|\lambda_i|$, $\rho(A)\leq||A||$
\item Theorem: If A is $n\times n$:
\begin{itemize}
    \item $||A||_2=[\rho(A^tA)]^{1/2}$
    \item $\rho(A) \le ||A||:\forall||\cdot||$
\end{itemize}
\item Convergent: $\lim_{k\to\infty}A^k=0$ $\Leftrightarrow$ $\rho(A)<1$
\item Matrix Norms: matrix norms have the following properties
  \begin{minipage}[t]{0.45\textwidth}
\begin{itemize}
    \item $||A||\geq 0$
    \item $||\alpha A || = |\alpha|\cdot ||A||$
    \item $||A+B|| \leq ||A|| + ||B||$
\end{itemize}
  \end{minipage}
\hfill
  \begin{minipage}[t]{0.45\textwidth}
\begin{itemize}
  \item $||A||=0 \leftrightarrow A=0$
  \item $||AB||=||A||||B||$
\end{itemize}
  \end{minipage}
\end{itemize}

\textbf{\S Iterative Methods}
\begin{itemize}
\item \textbf{General Iteration}: $x^{k+1}=Tx^k+c$
\item \textbf{Jacobi}: $x^{k+1}=D^{-1}(L+U)x^k+D^{-1}b$
\item \textbf{Gauss-Seidel}: $x^{k+1}=(D-L)^{-1}Ux^k+(D-L)^{-1}b$
\item \textbf{Stein-Rosenberg}: For matrices with positive diagonals: $\rho_{GS}\leq\rho_J<1$
\item \textbf{Speed of Convergence}: given matrices $T_{GS}=(D-L)^{-1}U$ and $T_J =D^{-1}(L+U)$, compare $\rho$. The bigger the $\rho$, the faster the convergence.
\item Error: $||x-x^k||\leq\frac{||T||^k}{1-||T||}||x^1-x^0||$
\item Stopping: $\frac{||x^k-x^{k-1}||}{||x^k||}<\varepsilon$
\end{itemize}

\textbf{\S2 Nonlinear Equations}
\begin{itemize}
\item \textbf{Bisection}:
\begin{itemize}
\item While $f(p_n)\neq0$ or $<T$: $p_n=\frac{a_1+b_1}{2}$
\item Error: $\frac{b_n-a_n}{2}<T$, $p=a+\frac{b-a}{2}$
\end{itemize}
\item \textbf{Fixed-Point}:
\begin{itemize}
\item $p_n=g(p_{n-1})$, converges if $|g'(x)|\leq K<1$
\item Algorithm: For $i<N_0$: $p=g(p_0)$, check $|p-p_0|<T$
\item A fixed point is defined as a point in which $p = f(p)$
\end{itemize}
\item \textbf{Newton}:
\begin{itemize}
\item $p_n=p_{n-1}-\frac{f(p_{n-1})}{f'(p_{n-1})}$
\item Quadratic convergence if $f'(p)\neq0$
\end{itemize}
\item \textbf{Secant}:
\begin{itemize}
\item $p_{n+1}=p_n-\frac{f(p_n)(p_n-p_{n-1})}{f(p_n)-f(p_{n-1})}$
\item Approx derivative: $\frac{f(p_{n-1})-f(p_{n-2})}{p_{n-1}-p_{n-2}}$
\item Algorithm: Store $q_0=f(p_0)$, $q_1=f(p_1)$, SET $p=p_1-\frac{q_1(p_1-p_0)}{q_1-q_0}$ IF STOPPING CONDITION: RETURN p; i++, $p_0=p_1, q_0=q_1, p_1=p, q_1=f(p)$ ENDWHILE OUTPUT FAILURE.
\end{itemize}
\end{itemize}

\textbf{\S Theorems}
\begin{itemize}
\item \textbf{Bisection}: Suppose $f\in C[a,b]:f(a)\cdot f(b) <0$. Bisection generates $\{p_n\}$ approximating a zero p with $|p_n-p|\le \frac{b-a}{2^n}:n\ge 1$
\item \textbf{Fixed Point}: If $g\in C[a,b]$, $g([a,b])\subseteq[a,b]$ g has a fixed point in $[a,b]$, additionally if $|g'|\leq K<1$, then the fixed point is unique.
\item \textbf{Fixed Point Theorem}: Let $g\in C[a,b]$ and $g(x)\in[a,b]:\forall x\in[a,b]$. Suppose as well that g' exists on (a,b) and positive $K<1$ exists with $|g'(x)|\le K:\forall x \in (a,b)$. Then for any number $p_0 \in [a,b]$h the sequence defined by $p_n=g(p_{n-1}):n\ge1$ converges to the unique point $p\in[a,b]$
\item \textbf {Corollary}: If g satisfies the hypothesis of the above theorem, $|p_n-p| \le k^n\max(p_0-a,b-p_0)$ and $|p_n-p|\le\frac{k^n}{1-k}|p_1-p_0|:\forall n>1$
\item \textbf{Newton}: For $f\in C^2[a,b]$ with simple root, $\exists\delta>0:p_0\in[p-\delta,p+\delta]$ converges.
\item \textbf{Matrix Invertibility}: $|A|\neq0$ $\Leftrightarrow$ unique solution $Ax=b$ $\Leftrightarrow$ $A^{-1}$ exists
\item \textbf{Taylor}: With $R_n(x)\Rightarrow f(x)=P_n(x)+R_n(x)$, $R_n(x)=\frac{f^{n+1}(\xi)}{(n+1)!}(x-x_0)^{n+1}$:$\xi \epsilon (x, x_0)$
\item \textbf{Existence of Inverse}: if A is square, $detA\ne 0 \leftrightarrow Ax=0$ has soln $x=0 \leftrightarrow Ax=b$ has a unique soln for any n-vector b. $\leftrightarrow A^{-1}$ exists.
\item \textbf{Diagonally Dominant Matrices}: dd matrices are nonsingular. A being dd means LU can be performed without P. A matrix is positive definite if $x^tAx>0$. PD matrices are nonsingular, $\forall i=1,\cdots,n: a_{ii} > 0$, $max|a_{kj}| > max|a_{ii}|$, $(a_{ij})^2<a_{ii}a_{jj}:\forall i\neq j$.
\item \textbf{Convergence and Spectral Radii}: equivalent statements: A is convergent, $\rho <1$, $lim A^nx=0:\forall x$, $lim ||A^n|| = 0 :\forall norms$.
\item \textbf{Convergence of DD}: If A is strictly DD, Jacobi and GS converge to the unique soln. Ax=b.
\item \textbf{Positive Definitive Check}: A matrix is PD if the leading principle submatrix determinants are positive.
\item \textbf{Positive Definitive Check}: A matrix is PD iff it may be factored into $LL^T$
\item \textbf{Non-singularity Check}: A matrix A has an inverse iff $\det A \ne 0$.
\item \textbf{Determinant of Triangular Matrices}: The determinant of a triangular matrix is $\Pi a_{ii}$.
\end{itemize}

\textbf{\S Proofs}
\begin{itemize}
    \item \textbf{Bisection (THM1)}: $\forall n\ge1$: $b_n-a_n=(b-a)\cdot\frac{1}{2^{n-1}} :p\epsilon (a_n, b_n)$. Since $p_n = \frac{1}{2}(a_n+b_n): \forall n \ge 1$, $|p_n-p|\le \frac{1}{2}(b_n-a_n)=\frac{b-a}{2^n}$. $\square$
    \item \textbf{Fixed Point}: Part i: If $g\in[a,b],g(x)\in[a,b]:\forall x\in[a,b]$ then g(x) has a fixed point in [a,b]:\\
    If $g(a)=a\text{ or } g(b)=b$, g has a fixed point at an endpoint. Suppose for contradiction that it does not. $g(a)>a \text{ and }g(b)<b$. Define $h(x)=g(x)-x$. Then h is cts on [a,b] and $h(a)=g(a)-a>0 \text{ and } h(b)=g(b)-b<0$ IVT states that $\exists p \in(a,b): h(p)=0$ Thus $g(p)-p=0 \Rightarrow$p is a fixed point of g.\\
    Part ii: Suppose as well $|g'(x)|\le k< 1:\forall x \in(a,b)$ and that $p,q\in[a,b]:p\ne q$. By MVT, $\exists \zeta: \frac{g(p)-g(q)}{p-q}=g'(\zeta)$. $|p-q|=|g(p)-g(q)|=|g'(\zeta)||p-q|\le k|p-q| < |p-q|$ contradiction.
\end{itemize}
\end{minipage}
\end{tcolorbox}
\newpage
\begin{minipage}[t]{0.49\textwidth}

\textbf{\S 2: Error Analysis and Acelerating Convergence}\\

\textbf{Basic Methods}
\begin{itemize}
    \item \textbf{Newton's Method}: Quadratic convergence if $f'(p)\neq0$. Iteration: \\$x_{n+1}=x_n-f(x_n)/f'(x_n)$.
    \item \textbf{Secant Method}: Superlinear convergence (order $\approx1.618$). Uses two previous points.
    \item \textbf{Newton's Improved Method}: $p_{n+1}=p_n-\frac{f(p_n)f'(p_n)}{f'(p_n)^2-f(p_n)f''(p_n)}$
\end{itemize}

\textbf{Convergence Analysis}
\begin{itemize}
  \item A soln p of $f(x)=0$ is a zero of multiplicity m of f if for $x\ne p$ we can write $f(x)=(x-p)^m q(x)$ where $\lim_{x\rightarrow p}{q(x)}\ne 0$. (Simple zeros are multiplicity 1). 
    \item Order of convergence $\alpha$: $\lim \frac{|p_{n+1}-p|}{|p_n-p|^\alpha}=\lambda$
    \item Linear ($\alpha=1$), Quadratic ($\alpha=2$)
    \item Fixed-point: Linear if $g'(p)\neq0$, quadratic if $g'(p)=0$ and $g''$ bounded.
\end{itemize}

\textbf{Special Cases}
\begin{itemize}
    \item Multiple roots: Modify Newton's using $\mu(x)=f(x)/f'(x)$
    \item \textbf{Aitken's $\Delta^2$}: Accelerates linear sequences. Is given by\\$\hat{p}_n = p_n - \frac{(p_{n+1}-p_n)^2}{(p_{n+2}-p_{n+1})-(p_{n+1}-p_n)}$
\end{itemize}

\textbf{Polynomial Methods}
\begin{itemize}
  \item Horner's method: Efficient evaluation ($n$ mults/adds) Algorithm: ex: evaluate $P(x)=\sum^{n}_{i=0}a_ix^i$ and derivative at $x_0$. Input $n,a_j, x_0: 0\le j\le n$. Output: $y=P(x_0)$, $z=P'(x_0)$. Set $y=a_n, z=a_n$. For $j=n-1,n-2,...,1$ set $y=x_0y+a_j,z=x_0z+y$. Endfor set $y-x_0y+a_0$ output y,z.  
    \item Deflation: Find roots sequentially via $P(x)\approx Q(x)(x-x_0)$
    \item Fundamental thm of alg: If $P(x)$ has a degree $n\ge 1$, $P(x)$ has at least one root.
    \item Cor: there also exists unique constants $x_1, ... x_k$ such that\\
    $\sum_{i=1}^{k}m_i=n, P(x)=a_n \times \prod_{i=1}^{k}(x-x_i)^{m_i}$
    \item Cor: these functions are unique.
\end{itemize}
\textbf{Weierstrass}: $\forall f$ cts on $[a,b]$, $\forall \varepsilon>0$, $\exists$ polynomial $p(x)$ with $|f(x)-p(x)|<\varepsilon$ $\forall x \in [a,b]$.

  \textbf{Lagrange Interpolation (unique!)}:\\ $P(x) = \sum_{m=0}^N f(x_m) L_m(x)$, where $L_m(x) = \prod_{\substack{k=0\\k\neq m}}^N \frac{x-x_k}{x_m-x_k}$

\textbf{Interpolation Error}: $f(x)-P(x) = \frac{f^{(n+1)}(\xi(x))}{(n+1)!} \prod_{k=0}^n (x-x_k)$ for $f\in C^{n+1}[a,b]$

\textbf{Newton's Divided Differences}:
\begin{align*}
f[x_i] &= f(x_i),\quad f[x_i,\dots,x_{i+k}] = \frac{f[x_{i+1},\dots,x_{i+k}] - f[x_i,\dots,x_{i+k-1}]}{x_{i+k}-x_i}
\end{align*}
$P_n(x) = \sum_{k=0}^n f[x_0,\dots,x_k] \prod_{j=0}^{k-1} (x-x_j)$

\textbf{Hermite}: Given $(x_j, f(x_j), f'(x_j))$, unique degree $\leq 2n+1$ we have:
  \begin{itemize}
\item $H(x) = \sum_{j=0}^n f(x_j)H_j(x) + \sum_{j=0}^n f'(x_j)\hat{H}_j(x)$
    \item $H_j(x)=[1-2(x-x_j)L_j'(x_j)]L_j^2(x)$
    \item $\hat{H}_j(x)=(x-x_j)L_j^2(x)$
    \item Note: $L_j(x)$ denotes the jth Lagrange coefficient polynomial of degree n. 
\item Error: $f(x)-H(x) = \frac{(x-x_0)^2\cdots(x-x_n)^2}{(2n+2)!}f^{(2n+2)}(\xi)$
\item Parametric curve Interpolation:
\begin{gather*}
    x(t)=[2(x_0-x_1)+3(a_0+a_1)]t^2+[3(x_1-x_0)-3(a_1+2a_0)]t^2+3a_0t+x_0\\
    y(t)=[2(y_0-y_1)+3(\beta_0+\beta_1)]t^3+[3(y_1-y_0)-3(\beta_1+2\beta_0)]t^2+3\beta_0t+y_0
    \end{gather*}
  \end{itemize}
  \textbf{Cubic Splines}:\\
  \begin{minipage}[t]{0.45\textwidth}
    \begin{itemize}
      \item{$S_j^{(n)}(x)=S_{j+1}^{(n)}(x): n=0,1,2; x$ is a boundary point }
      \item{Err: $\max{|f(x)-S(x)|}\le \frac{5M}{384}h^4:h=\max(x_{j+1}-x_j),$\\$M=\max|f^{(4)}|$}
    \end{itemize}
  \end{minipage}
  \hfill
  \begin{minipage}[t]{0.45\textwidth}
    \begin{itemize}
      \item $S(x_j)=f(x_j):\forall j$ provided. \\$S\in C^2[a,b]$ 
      \item{Clamped: $S'(a)=f'(a), S'(b)=f'(b): a,b $ are endpoints }
      \item{Natural: $S''(a)=S''(b)=0:a,b$ are endpoints }
    \end{itemize}
  \end{minipage}

\textbf{Richardson Extrapolation}: $N_{j+1}(h) = N_j(h/2) + \frac{N_j(h/2) - N_j(h)}{2^{j}-1}$

\textbf{Numerical Integration}:
\begin{itemize}\setlength{\itemsep}{-1pt}
\item \textbf{Trapezoid Rule} ($O(h^2)$): 
    \begin{itemize}\setlength{\itemsep}{-2pt}
        \item Single: $\int_a^b f(x)\,dx \approx \frac{h}{2}[f(a) + f(b)]$, Error: $-\frac{h^3}{12}f''(\xi)$
        \item Composite: $\int_a^b f(x)\,dx \approx \frac{h}{2}[f(a) + 2\sum_{i=1}^{n-1}f(x_i) + f(b)]$
        \item Error: $-\frac{(b-a)h^2}{12}f''(\xi) \approx -\frac{M(b-a)^3}{12n^2}$ where $M=\max|f''|$
    \end{itemize}

\item \textbf{Midpoint Rule} ($O(h^2)$):
    \begin{itemize}\setlength{\itemsep}{-2pt}
        \item Single: $\int_a^b f(x)\,dx \approx (b-a)f(\frac{a+b}{2})$
        \item Composite: $\int_a^b f(x)\,dx \approx h\sum_{i=1}^{n}f(a+(i-\frac{1}{2})h)$
    \end{itemize}

\item \textbf{Simpson's Rules} ($O(h^4)$):
    \begin{itemize}\setlength{\itemsep}{-2pt}
        \item 1/3 Rule: $\int_a^b f(x)\,dx \approx \frac{h}{3}[f(a) + 4f(\frac{a+b}{2}) + f(b)]$
        \item Composite: $\int_a^b\!f(x)\,dx \approx \frac{h}{3}[f(a) + 4\sum_{i=1,3,5}^{n-1}\!f(x_i) + 2\sum_{i=2,4,6}^{n-2}\!f(x_i) + f(b)]$
        \item Error: $-\frac{h^5}{90}f^{(4)}(\xi)$ (single), $-\frac{(b-a)h^4}{180}f^{(4)}(\xi)$ (composite)
        \item 3/8 Rule: $\int_a^b\!f(x)\,dx \approx \frac{3h}{8}[f(x_0)+3f(x_1)+3f(x_2)+2f(x_3)+\ldots+f(x_n)]$
        \item 3/8 Error: $-\frac{(b-a)^5}{6480}f^{(4)}(\xi)$
    \end{itemize}
\end{itemize}

\textbf{Romberg - $O(h_k^{2j})$}: 
\begin{itemize}
    \item{$R_{k,j} = R_{k,j-1} + \frac{R_{k,j-1} - R_{k-1,j-1}}{4^{j-1}-1}$, error $O(h^{2j})$}
    \item{$R_{k,1}$ represents the approximation of the integral using $m_k=2^{k-1}$ intervals}
\end{itemize}

\textbf{\S ODE Initial Value Problems}
\begin{itemize}
\item \textbf{Basic Problem}: $y'(t) = f(t,y)$, $y(a) = \alpha$
\item \textbf{Lipschitz Condition}: $|f(t,y_1) - f(t,y_2)| \leq L|y_1 - y_2|$ 
\item Existence/uniqueness guaranteed when $|\frac{\partial f}{\partial y}| \leq L$ over convex domain $D$
\end{itemize}

\textbf{Numerical Methods}
\begin{itemize}
\item \textbf{Euler's Method}: $w_{i+1} = w_i + hf(t_i,w_i)$ [Error: $O(h)$]
\item \textbf{Taylor Methods}: $w_{i+1} = w_i + hT^{(n)}(t_i,w_i)$ where 
$T^{(n)} = f + \frac{h}{2}f' + \cdots + \frac{h^{n-1}}{n!}f^{(n-1)}$
\item \textbf{Runge-Kutta Methods}:\\
\begin{minipage}[t]{0.45\textwidth}
    \textbf{Midpoint (RK2)}\\
    $w_{i+1} = w_i + hf(t_i+\frac{h}{2}, w_i+\frac{h}{2}f_i)$
\end{minipage}
\hfill
\begin{minipage}[t]{0.45\textwidth}
    \textbf{Modified Euler (RK2)}\\
    $w_{i+1} = w_i + \frac{h}{2}(f_i + f(t_{i+1},w_i+hf_i))$
\end{minipage}

\item \textbf{Classical RK4}:
$\begin{array}{l}
k_1 = hf(t_i,w_i) \\
k_2 = hf(t_i+\frac{h}{2},w_i+\frac{k_1}{2}) \\
k_3 = hf(t_i+\frac{h}{2},w_i+\frac{k_2}{2}) \\
k_4 = hf(t_i+h,w_i+k_3) \\
w_{i+1} = w_i + \frac{1}{6}(k_1 + 2k_2 + 2k_3 + k_4)
\end{array}$
[Error: $O(h^4)$]
\end{itemize}
\end{minipage}
\hfill
\begin{minipage}[t]{0.49\textwidth}
    \textbf{Error Analysis}
    \begin{itemize}
    \item \textbf{Local truncation error}: $\tau_{i+1} = y(t_{i+1}) - w_{i+1}$ given $w_i = y(t_i)$
    \item \textbf{Global truncation error}: Accumulated error across all steps
    \item For Euler: $|\tau_i| \leq \frac{h^2}{2}M$ (local), $O(h)$ (global) where $M = \max|y''|$
    \item For RK4: $O(h^5)$ (local), $O(h^4)$ (global)
    \end{itemize}
    
    \textbf{Stability \& Step Size}
    \begin{itemize}
    \item Well-posed problem requires: unique solution exists + small input changes → small output changes
    \item Step size formula: $h < \frac{2\varepsilon}{M(b-a)}$ for error $\varepsilon$, where $M = \max|y''|$
    \item Example: $y'=y\cos t$ has Lipschitz constant $L=1$ since $|\frac{\partial f}{\partial y}| = |\cos t| \leq 1$
    \end{itemize}
\textbf{Chapter 3: Interpolation}
\textbf{Lagrange Interpolation}
\begin{itemize}
    \item \textbf{$\Delta^k$}: $\Delta^2f_i=\Delta(\Delta(f_i))=\Delta(f_{i+1}-f_i)=f_{i+2}-2f_{i+1}+f_i$
    \item Reuse computations with \textbf{Neville's Method}:\\ for $i=1,2,...,n$ do: for $j=1,2,...,i$ do $Q_{i,j} \leftarrow \frac{(x-x_{i-j}Q_{i,j-1})-(x-x_i)Q_{i-1,j-1}}{x_i-x_{i-j}}$
\end{itemize}

\textbf{Hermite Interpolation}
\begin{itemize}
    \item Error term: \( \frac{(x-x_0)^2\cdots(x-x_n)^2}{(2n+2)!}\max|f^{(2n+2)}| \)
    \item Handle derivatives via \textbf{divided differences} with repeated nodes: \( z_{2i}=z_{2i+1}=x_i \)
    \item \textit{Warning}: Noisy derivatives \(\Rightarrow\) amplified errors.
\end{itemize}

\textbf{Splines}
\begin{itemize}
    \item \textbf{Cubic splines}: Solve tridiagonal system (\(O(n)\) ops) with:
        \[ h_{i-1}c_{i-1} + 2(h_{i-1}+h_i)c_i + h_ic_{i+1} = \frac{3}{h_i}(a_{i+1}-a_i) - \frac{3}{h_{i-1}}(a_i-a_{i-1}) \]
    \item \textbf{Natural splines} (\(S''=0\)): Stable but less accurate.  
    \textbf{Clamped splines}: Require \(f'(a), f'(b)\) but higher accuracy.
    \item Error: \( O(h^4) \) for \( f \in C^4 \), \( O(h^2) \) for linear splines.
\end{itemize}

\textbf{Trapezoidal \& Simpson's Rules}
\begin{itemize}
    \item \textbf{Romberg Integration}: Accelerate Trapezoidal Rule via:
        $R_{k,j} = R_{k,j-1} + \frac{R_{k,j-1}-R_{k-1,j-1}}{4^{j-1}-1}$
    \item Stop when $|R_{k,k}-R_{k-1,k-1}| < \epsilon$.
    \item The degree of precision of a quadrature formula is the largest n such that the formula is exact for $x^k: \forall k=0,1,\cdots,n$
\end{itemize}

\textbf{Adaptive Quadrature}
\begin{itemize}
    \item Error estimate: \( \frac{1}{15}|S(a,b)-S(a,c)-S(c,b)| \), \( c=(a+b)/2 \)
    \item Subdivide intervals where error \( > \epsilon/2 \).
    \item Step 1: Apply Simpson's with $h=(b-a)/2$. \\$\int_a^bf(x) = h/3\left[f(a)+4f(a+h)+f(b)\right]-\frac{h^5}{90}f^{(4)}(\mu) \mu\in(a,b)$
    \item Step 2: Find error using Simpson's on $h=(b-a)/4$\\
    $\int_a^bf(x)=h/6[f(a)+4f(a+h/2)+2f(a+b)+4f(a+3h/2)+f(b)]-(\frac{h}{2})^4\cdot \frac{(b-a)}{180}f^{(4)}(\tilde{\mu}): \tilde{\mu}\in(a,b)$
    \item Note: We assume $f^4(\mu)=f^4(\tilde{\mu})$: true for small h.
    \item Step 3: Calculate error as $1/10\left| S(a,b)-S(a,\frac{a+b}{2})-S(\frac{a+b}{2},b)\right| < \varepsilon $
    \item Step 4: If true, RETURN. Else, GOTO step 1.
\end{itemize}

\textbf{Gaussian Quadrature}
\begin{itemize}
  \item $\int_a^b w(x)f(x)dx=\sum_{i=1}{n}w_if(x_i)$ where $w(x)$: weight functions, $w_i$: weight at i, $x_i$ node at i. 
    \item Nodes: Roots of Legendre polynomials \( P_n(x) \). Weights: \( c_i = \int_{-1}^1 \prod_{j\neq i}\frac{x-x_j}{x_i-x_j}dx \)
    \item Exact for polynomials of degree \( \leq 2n-1 \).

    \item Step 1: Transform to \([-1,1]\): \( x = \frac{(b-a)t + (a+b)}{2} \), \( dx = \frac{b-a}{2}dt \)
    \item Step 2: substitute x into integrand
    \item Step 3: use the formulae to get the answer:
    \begin{itemize}
      \item 1-point: $\int_{-1}^1f(x)dx=2f(0)$
      \item 2-point: $\int_{-1}^1f(x)dx=f(-\frac{1}{\sqrt{3}})+f(\frac{1}{\sqrt{3}}): w_1=w_2;x_1=-\frac{1}{\sqrt{3}}, x_2 = \frac{1}{\sqrt{3}}$
      \item 3-point: $\int_{-1}^1f(x)dx=\frac{5}{9}f(-\sqrt{\frac{3}{5}})+\frac{8}{9}f(0)+\frac{5}{9}f(\sqrt{\frac{3}{5}})$
    \end{itemize}
\end{itemize}

\textbf{Legrendre Polynomials}
\begin{itemize}
    \item{$P_n(x)$ denotes the n'th degree Legrendre polynomial } 
    \item{$\int_{-1}^1 P(x)P_n(x)dx = 0: $ P(x) is of degree < n}
    \item{$P_0(x)=1, P_1(x)=x, P_2(x) = x^2-1/3, P_3(x)=x^3-3/5x, P_4(x)=x^4-6/7x^2+3/35$}
  \end{itemize}

\textbf{Tips:}
\begin{itemize}
  \item{Use Gaussian quadrature with $n=c\rightarrow$ apply c-point Gauss-Legendre Rule.}
  \item{When asked for the degree of precision, plug in $x^0, x^1, x^2, \cdots, x^n$ until failure. The n before it fails is the degree of precision}
\end{itemize}

\textbf{Theorems:}
\begin{itemize}
  \item {\textbf{Thm 2.8:} Let p be a soln of the eq $x=g(x)$ and suppose $g'(p)=0$ and $g''$ is cts and strictly bounded by M on an interval I containing p. Then $\exists \delta > 0$ such that $p_0\in[p-\delta, p+\delta]$ the seq: $p_n=g(p_{n-1}): n\ge1$ converges at least quadratically to p. Moreover, for large n, $|p_{n+1}-p|<\frac{M}{2}|p_n-p|^2$ }
  \item {\textbf{Thm 2.10:} $f\in C'[a,b]$ has a simple zero at p in (a,b) iff $f(p)=0,f'(p)\ne0$ }
  \item {\textbf{Thm 2.11:} The function $f\in C^m[a,b]$ has a zero of multiplicity m at p iff $0=f(p)=f'(p)=f''(p)=\cdots=f^{(m-1)}(p)$ }
  \item {\textbf{Thm 3.3}: Suppose $x_0,x_1,...,x_n$ are distinct numbers in [a,b] and $f\in C^{n+1}[a,b]$ then for each x in $[a,b]$, a number $\xi(x)$ in (a,b) exists with}\\
  $f(x)=P(x)+\frac{f^{(n+1)}(\xi(x))}{(n+1)!}\prod_{i=0}^{n}(x-x_i)$
  with P(x) being the nth Lagrange interpolating polynomial
  \item {\textbf{Thm: Err Trapezoid}: Let $f\in C^2[a,b]$, $h=(b-a)/n$, $x_j=a+jh:0\le j\le n$. Then $\exists\mu\in(a,b)$ for which the composite trapezoid rule with n subivls has an err term of $\frac{b-a}{12}h^2f''(\mu)$}
  \item {\textbf{Thm: Legrendre Thm}}: suppose $x_1,x_2,\cdots,x_n$ are the roots of the nth degree Legrendre Polynomial and $\forall i = 1,2,\cdots,n$ are the numbers $c_i$ such that $c_i=\int_{-1}^1\prod_{j=1,j\ne}^n \frac{x-x_j}{x_i-x_j}dx$. If $P(x)$ is any polynomial of degree $<2$, then $\int_{-1}^1P(x)dx=\sum_{i=1}^nc_iP(x_i)$ 
\end{itemize}


\end{minipage}

\end{document}
